\documentclass{article}
\usepackage{../fasy-hw}
\usepackage{ wasysym }

%% UPDATE these variables:
\renewcommand{\hwnum}{1}
\title{Discrete Structures, Homework 1}
\author{Robert Marsh(JamesBean\#0678)}
\date{due: 22 January 2021}
\collab{n/a}
\begin{document}

\maketitle

This homework assignment should be
submitted as a single PDF file both to D2L and to Gradescope.

General homework expectations:
\begin{itemize}
    \item Homework should be typeset using LaTex.  (Note: if you are still
        having trouble with your setup, please reach out to the instructor and
        TA).
    \item Answers should be in complete sentences and proofread.
    \item You will not plagiarize.
    \item List collaborators at the start of each question using the
        \texttt{collab} command.
    \item Put your answers where the \texttt{todo} command currently is (and
        remove the \texttt{todo}, but not the word \texttt{Answer}).
\end{itemize}

% ============================================
% ============================================
\collab{\small{Patrick OConnor, Chandler Norby, Patrick Schnabel, Silas Andrews, Avery Jacobson}}
\nextprob{Getting to Know Your Classmates}
% ============================================
% ============================================

Find a different classmate for each of the following:
\begin{enumerate}
    \item Was born in the same month as you (year can be different).
        \paragraph{Answer} 
        I was born in September, as was Patrick OConnor.

    \item Has a shared hobby with you.
        \paragraph{Answer} 
        Chandler Norby and I share video games as a hobby.

    \item Has the same middle initial as you.
        \paragraph{Answer} 
        Patrick Schnabel and I share the middle initial "J".

    \item Lives in a different building than you.
        \paragraph{Answer} Silas Andrews lives off campus East of the Hospital, I live off campus next to Lowe's.

    \item Has eaten at at least one restaurant or traveled to at least one city that you have not been
        (yet).
        \paragraph{Answer} Avery Jacobson has eaten somewhere I haven't: Hooters, in Long Beach, California.

\end{enumerate}

% ============================================
% ============================================
\collab{n/a}
\nextprob{Why Proofs?}
% ============================================
% ============================================

Much of this class is spent learning how to prove things.  Explain why it is
important to you, as a computer scientist, to know how to prove things
mathematically.

\paragraph{Answer}

In both academic and professional context, computer scientists construct, implement, and test code with a specific function in mind. For most projects it is far too time consuming to experimentally demonstrate that the code will work for all relevant inputs. Knowing how to prove things mathematically allows computer scientists to implement code that they know will work for a specific set of data, by proving it through mathematic theory instead. This applies to both the base workability of the code solution, and the efficiency compared to other solutions. There are significant confounding variables when testing the run time of two different algorithms; proving it mathematically can be a more reliable method.


% ============================================
% ============================================
\collab{Professor Fasy (Lecture + Discord guidance on how to do this proof)}
\nextprob{A Proof}
% ============================================
% ============================================

Prove that $6\Z \subset 2\Z$.

\paragraph{Answer} First, we must define $6\Z$ and $2\Z$:

$2\Z := \{ n \in \Z $ $|$ $ \exists $ $y \in \Z $ $s.t.$ $n=2y \}$

$6\Z := \{ n \in \Z $ $|$ $ \exists $ $x \in \Z $ $s.t.$ $n=6x \}$

Let $n \in 6\Z$.  By definition of $6\Z$ :
\begin{align}
    \nonumber n & = 6m \text{ where } m \text{ is an integer} \\
    \nonumber n & = 2(3m) && \text{\emph{through algebra }}
\end{align}
Let $x = 3m$. Through substitution, $n=2x$. By definition of $2\Z$, $n\in 2\Z$, as was to be shown.


% ============================================
% ============================================
\collab{https://www.britannica.com/biography/Grace-Hopper}
\nextprob{Grace Hopper}
% ============================================
% ============================================



\paragraph{Answer}

Grace Hopper was a computer scientist during the mid 20th century. After her schooling at Vassar and Yale, she worked on the United States' first government computer projects, the "Mark I", which by today's standards would be considered a calculator. She wrote the manual for this first "computer", she wrote the first compiler, she helped design the UNIVAC I (one of the first commercial business computers, also a calculator by today's standards). On top of all of this general work to the development of computing projects, her greatest contribution was the development of \emph{compilers}. This allowed programmers to write instructions in the English language, and later translate them into machine code. This abstraction greatly increases the complexity of problems that can be solved with computer programs, as the time and thinking power can be spent on broader concepts without getting bogged down in the details of implementing them step by step, for every individual calculation on a specific piece of hardware.

% ============================================
% ============================================
\collab{\todo{}}
\nextprob{Bonus Question!}
% ============================================
% ============================================

Use the `figure` environment to add a figure that provides the solution to
Exercises set 1.4, Problem 4.  Your figure can be hand drawn and scanned, or can
be made using a tool such as Inkscape.

\paragraph{Answer}

\todo{whenever you add a figure, be sure to add a reference to the figure too!
(otherwise, the reader might forget to look at the figure)}

\end{document}

