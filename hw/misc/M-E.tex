\documentclass{article}
\usepackage{../fasy-hw}
\usepackage{ wasysym }

%% UPDATE these variables:
\title{Discrete Structures, Homework 1}
\author{Robert Marsh(JamesBean\#0678)}
\date{due: 22 January 2021}
\collab{n/a}
\begin{document}

\maketitle

This homework assignment should be
submitted as a single PDF file both to D2L and to Gradescope.

General homework expectations:
\begin{itemize}
    \item Homework should be typeset using LaTex.  (Note: if you are still
        having trouble with your setup, please reach out to the instructor and
        TA).
    \item Answers should be in complete sentences and proofread.
    \item You will not plagiarize.
    \item List collaborators at the start of each question using the
        \texttt{collab} command.
    \item Put your answers where the \texttt{todo} command currently is (and
        remove the \texttt{todo}, but not the word \texttt{Answer}).
\end{itemize}

% ============================================
% ============================================
\clearpage

\setlength{\parindent}{10ex}
\par

From this lecture I learned that System Health is a field with people actively working in it. Previously, I had only read about people studying these problems in a very narrow scope: e.g. tales of specific businesses and people working to solve a specific set of problems for their work. This is the first time I’ve learned about the specific set of tools and procedures for solving any general problem. I’ve held a strong fascination in this subject for years, through my work for a small contractor and as a coach for a few different clubs - but it’s awfully hard to research it without knowing the industry name.
\par The POINTER (Portable Interactive Troubleshooter) - step by step troubleshooting, while offering explanations to teach the operator how the problem was recognized. Later they created “Smart” Testers, for automated troubleshooting without operator guidance. Specialized testers were later removed in the 1990s due to an effort for standardization by the Department of Defense. As I understand it, the goal was one tester for the entire group of products from a single manufacturer, rather than one for each product.
\par The latest problems that Professor Sheppard has been working on are predictive diagnostics, using probabilistic models of diagnostics. In addition, vehicle level reasoning system for NASA, and work with the Navy to combine data-driven and physics based models into a single result: the Standards-based Analysis Platform for Predictive Health Integrated Reasoning Environment (SAPPHIRE). The latest push has been to find a way to use risk-informed decision making in other industries, such as for food and agriculture. And of course, AI and Machine Learning.
\par This also caused me to go out and research “AI Winter” - a topic I’d never heard before. I didn’t know that the development of AI, or even computer calculating systems, had waxed and waned so much in its funding and public interest since the 1950s. Now that I looked up a timeline of it, the “AI Summers” line up quite perfectly with the major computer science feats that I’ve learned about in other areas.

This lecture relates to our class through consideration of probability, algorithm analysis, and even proof of algorithms. Much of Professor Sheppard’s work in the last 20 years has been primarily focused on probability-based prediction, in contrast to his earlier work a la POINTER with a focus on evidence-based troubleshooting. Probability-based predictions rely on the accuracy of their probabilistic models, and the algorithms that operate within them. From my understanding, much of Machine Learning and AI at this stage is fine-tuning specialized algorithms to obtain desired outputs, or “algorithms with tricks” as Prof. Sheppard calls them. The fine-tuning and model-based predictions all rely on probabilistic calculations, through specialized algorithms - two big topics in the latter half of this semester! 
\par In addition, the “proof” of the probability-based algorithms used for this decision making reminds me quite closely of the late concerns in the Four Colors Suffice regarding the verifiability of the Four Color Theorem (FCT). The mathematical “purists” opposed the evidence-based proof by exhaustion due to concerns over the reliability of the methods used for exhaustion (the computer). In a similar way, one could raise the concern over the lack of proof for these models. They create results which are “likely” to be true, but not in an exhaustive or 100% reliable manner. Sure, the algorithms which calculate the probabilities might be provable, but it raises a question of safety and reliability for diagnosis with this method (which I’m sure is taken into consideration).
\par I found the interdisciplinary fashion of his work to be particularly interesting. In his own words, Professor Sheppard comes from a computer science background with a focus on maths. In order to help the particular engineering fields with which he was working, he had to go in and build or learn a common language with them before they could advance. The “computer scientist learning engineer” here is in bright contrast with the typical pop culture story motif of people engineering fields learning something about computer science to advance their work. I wonder, then, whether this story will change to follow more like Professor Sheppard’s timeline, as Computer Science ages as a field and develops a deeper backlog of knowledge.
\par I found Professor Sheppard’s comments on his personal perspective regarding his work in AI to highlight him as someone that I could see as a mentor. Near the end, he mentioned that he works with the military to develop AI not because he supports war, but because he wants to make sure that it’s done safely, ethically, and reliably. What I inferred from this is that it would be nigh impossible to prevent such research being done, and by contributing to it himself Prof. Sheppard can be a part of the conversation and decision making process in a way that ensure that the work is done safely. I think that Prof. Sheppard’s active steps to shepherd the development of new technology in a positive direction demonstrates a strong sense of ethics for the field. I’m deeply interested in the field of law and the potential impact of new technology on that field, and think that Prof. Sheppard would have an interesting perspective to share on that field as well.

% ============================================
% ============================================



\end{document}

